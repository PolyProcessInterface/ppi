\documentclass{article}
\usepackage{graphicx}

\date{\today}
\author{Tarik Atlaoui \\ Nicolas Peugnet \\ Kimmeng Ly \\ Max Eliet}

\begin{document}


\begin{titlepage}
	\enlargethispage{2cm}
	\newcommand{\HRule}{\rule{\linewidth}{0.5mm}}
	\center
	\textsc{\LARGE
	Pre-rapport du PSAR 
	} \\[1cm]
	\HRule \\[0.4cm]
	{ \huge \bfseries API générique pour le développement d'applications réparties \\[0.15cm] }
	\HRule \\[4cm]
	\large{Tarik Atlaoui \\[3mm] Nicolas Peugnet \\[3mm] Kimmeng Ly \\[3mm] Max Eliet} \\[3cm]
	09 Mars 2020 \\[3cm]
	\hfill \includegraphics[width=5cm]{logoSU.jpg}
\end{titlepage}

	\newpage
	\pagenumbering{arabic}
		\section{Introduction}
			\large{
			\indent Notre projet a pour but d'implémenter un intergiciel permettant l'execution d'applications réparties, aussi bien sur une infrastructure réelle que sur un simulateur, sans avoir à en modifier le code, afin de simplifier le développement de futures applications réparties.
\\[2mm]
			 \indent Dans notre cas, nous avons choisi d'implémenter cet intergiciel pour MPI (Message Passing Interface) en tant qu'API d'infrastructure réelle, et PeerSim en tant que simulateur à événements discrets.}

		\section{Cahier des charges}
			\subsection{Choix d'une interface générique}
				\large{ \indent Nous avons commencé par définir une API générique : \textit{Infrastructure} regroupant les fonctionnalités qu'offriront à la fois PeerSim et MPI aux applications réparties, à commencer par une primitive d'envoi de message, que nous enrichirons au fur et à mesure.\\[2mm]
					\indent De plus, nous avons encapsulé les fonctions de chaque noeud du système dans une classe \textit{NodeProcess}, par exemple le traitement à effectuer lors du démarrage du noeud ou de la récepion d'un message.  }

			\subsection{Implantation de l'interface pour PeerSim et MPI}
				\large{ Nous avons ensuite implanté les premières fonctionnalités de notre interface nécessaires au fonctionnement d'un programme de test basique faisant circuler d'un messsage à travers un anneau de noeuds du système.}

			\subsection{Création d'une classe \textit{Message}}
				\large{Nous avions ensuite besoin d'une classe représentant l'enveloppe d'un message, qui sera étendue par l'utilisateur pour ses propres messages.}

	\newpage
			\subsection{Unification du lancement de l'application}
				\large{Ensuite, nous avons souhaité unifier et simplifier le lancement de l'application par le biais d'une interface /textit{Runner} avec une implantation pour MPI et PeerSim, afin que le lancement d'une application se résume à choisir entre celles-ci.}
			\subsection{Factorisation de l'aiguillage des messages}
				\large{La fonctionnalité suivante que nous avons ajouter est celle de l'aiguillage automatique d'un message vers le traitement qui lui est adéquat, à l'aide d'annotations.}
			\subsection{Description d'un scenario}
				\large{La dernière fonctionnalité que nous avons ajouté jusqu'à présent nous permet de décrire un scénario, correspondant à l'execution d'une application, sous la forme d'un fichier JSON.}
			\subsection{Fonctions de wait et notify}
				\large{La prochaine fonctionnalité sur laquelle nous travaillons est celle de l'attente d'un processus sur une condition ou un message particulier.}
			\subsection{Autres fonctionnalités}
				\large{Nous pourrons ajouter d'autres fonctionnalités qui nous viendraient à l'esprit, si le temps le permet.}
\end{document}
