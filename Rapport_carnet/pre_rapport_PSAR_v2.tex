\documentclass{article}
\usepackage{graphicx}

\date{\today}
\author{Tarik Atlaoui \\ Nicolas Peugnet \\ Kimmeng Ly \\ Max Eliet}

\begin{document}


\begin{titlepage}
	\enlargethispage{2cm}
	\newcommand{\HRule}{\rule{\linewidth}{0.5mm}}
	\center
	\textsc{\LARGE
	Pre-rapport du PSAR 
	} \\[1cm]
	\HRule \\[0.4cm]
	{ \huge \bfseries API générique pour le développement d'applications réparties \\[0.15cm] }
	\HRule \\[4cm]
	\large{Tarik Atlaoui \\[3mm] Nicolas Peugnet \\[3mm] Kimmeng Ly \\[3mm] Max Eliet} \\[3cm]
	09 Mars 2020 \\[3cm]
	\hfill \includegraphics[width=5cm]{logoSU.jpg}
\end{titlepage}

	\newpage
	\pagenumbering{arabic}
		\section{Introduction}
			\subsection{Les applications réparties : qu'est-ce ?}
			 \subsection{Les difficultés à programmer une application répartie}

		\section{Méthode de développement}
			\subsection{API réelle : avantages/inconvénients + exemple API(MPI ou autre)}
			\subsection{API simulation à évenement discret : qu'est-ce ? + avantages/inconvénients + exemple(PeerSim ou autre)}
			\subsection{Code des deux API faisant la même chose}
		
		\section{Motivation et objectif global}
			\subsection{API générique : avantages, modèle de programmation(ici événementiel)}
			\subsection{Code précédent en version générique}
		
		\section{Les étapes de réalisation}
			\subsection{Définition API générique}
			\subsection{Définition des primitives offertes et explication}
			\subsection{Implantation vers MPI}
			\subsection{Implantation vers PeerSim}
			\subsection{API pour un scenario}
			\subsection{Rédaction du rapport}

		\section{Plan de validation}
			\subsection{Comment montrer que cela fonctionne bien, tests unitaires, respect du cahier des charges}
		\section{Planning des tâches}
			\subsection{Une deadline pour chaque tâche dans un ordre chronologique}
\end{document}
